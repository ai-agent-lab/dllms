Sampling strategies are central to the trade‐off between quality, speed, and stability in diffusion‐based text generation. In \textbf{energy‐based diffusion models}, a pretrained autoregressive (AR) model serves as an energy function to reweight samples drawn from the diffusion proposal distribution. The method restricts the importance sampling to a window of late diffusion timesteps and significantly reduces parallel decoding error. It enables high-fidelity outputs with significantly fewer denoising iterations, thereby reducing overall wall-clock time. Complete MCMC sampling remains infeasible in the high-dimensional token space. Therefore, importance sampling remains the preferred alternative \cite {xu_energy-based_2025}.  



Furthermore, masked diffusion models can accelerate inference by skipping redundant noise levels. In “\textbf{Simple and Effective Masked Diffusion Language Models},” the authors introduce an \textbf{Efficient Ancestral Sampling} scheme that dynamically omits specific intermediate timestamps during denoising, reducing the number of functional calls to the denoiser without degrading output quality \cite{sahoo_simple_2024}.  



Recently, the \textbf{Large Language Diffusion Models (LLaDA)} framework employs two complementary remasking strategies to concentrate computation on uncertain tokens. First, \emph{low‐confidence remasking} re‐noises only those tokens whose predicted confidence falls below a threshold, focusing denoising steps where they matter most. Second, a \emph{semi‐autoregressive remasking} divides the sequence into blocks that are generated left‐to‐right but sampled in parallel within each block, achieving a balance between sequential coherence and parallel throughput \cite{nie_large_2025}.  



Generalized Interpolating Discrete Diffusion (GIDD) adds a self‐correction iteration after full denoising: the model resamples tokens based on its likelihood estimates, committing the highest‐scoring changes until convergence. The fixed-point refinement step mitigates error propagation from early timesteps, thus yields more coherent sequences without requiring additional training \cite {rutte_generalized_2025}.  



To sum up, these sampling innovations: importance sampling windows, efficient ancestral skipping, confidence-based remasking, and self-correction demonstrate the careful scheduler and sampler design can significantly improve both the speed and quality of diffusion-based text generation.  



% \bibliographystyle{plain}

% \bibliography{DLLM}

